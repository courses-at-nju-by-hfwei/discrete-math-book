% 3-induction-theory.tex

%%%%%%%%%%%%%%%%%%%%%%%%%%%%%%
\section{数学归纳法与良序原理} \label{section:induction-theory}

\begin{theorem}{第一数学归纳法}{first-induction}
  设 $P(n)$ 是关于自然数的一个性质。如果
  \begin{enumerate}[(i)]
    \setlength{\itemsep}{8pt}
    \item $P(0)$ 成立;
    \item 对任意自然数 $n$, 如果 $P(n)$ 成立, 
      则 $P(n+1)$ 成立。
  \end{enumerate}
  那么, $P(n)$ 对所有自然数 $n$ 都成立。
\end{theorem}

可以表达为推理规则:
\[
  \nd{P(0) \qquad \forall n \in \N.\;
  \Big(P(n) \to P(n+1) \Big)}{\forall n \in \N.\; P(n)}{\text{第一数学归纳法}}
\]

或表达为一阶谓词逻辑公式:
\[
  \biggl(P(0) \land \forall n \in \N.\; \Big(P(n) \to P(n+1) \Big) \biggr) \to \forall n \in \N.\; P(n).
\]

\begin{remark}
  TODO: 一阶 vs. 高阶
\end{remark}

\begin{theorem}{第二数学归纳法 (The Second Mathematical Induction)}{second-induction}
  设 $Q(n)$ 是关于自然数的一个性质。
  如果
  \begin{enumerate}[(i)]
    \setlength{\itemsep}{8pt}
    \item $Q(0)$ 成立;
    \item 对任意自然数 $n$, 如果 $Q(0), Q(1), \dots, Q(n)$ 都成立, \\ 则 $Q(n+1)$ 成立。
  \end{enumerate}
  那么, $Q(n)$ 对所有自然数 $n$ 都成立。
\end{theorem}

\[
  \nd{Q(0) \qquad \forall n \in \N.\; \Big(\big(Q(0) \land \dots \land Q(n)\big) \to Q(n+1) \Big)}{
    \forall n \in \N.\; Q(n)}{\text{第二数学归纳法}}
\]

\[
  \biggl(Q(0) \land \forall n \in \N.\; \Big(\big(Q(0) \land \cdots \land Q(n)\big) \to Q(n+1) \Big) \biggr)
    \to \forall n \in \N.\; Q(n).
\]

\begin{theorem}{}{second-implies-first}
  第二数学归纳法蕴含第一数学归纳法。
\end{theorem}

\begin{theorem}{}{first-implies-second}
  第一数学归纳法蕴含第二数学归纳法。
\end{theorem}

\begin{proof}
  令 $P(n) \triangleq Q(0) \land \ldots \land Q(n)$。
  第一数学归纳法对上述 $P(n)$ 也成立, 即
  \begin{align}
    \biggl(Q(0) \land \forall n \in \N.\;
    \Big(Q(0) \land \ldots \land Q(n) \to
      Q(0) \land \ldots \land Q(n+1) \Big) \biggr)
    \to \forall n \in \N.\; Q(0) \land \ldots Q(n).
    \label{eqn:first-induction}
  \end{align}
  要证明第二数学归纳法, 先引入如下两个前提
  \begin{align}
    Q(0) \label{eqn:second-induction-base} \\
    \forall n \in \N.\; Q(0) \land \ldots Q(n) \to Q(n+1)
  \end{align} \label{eqn:second-induction}
  首先, 
  \begin{align}
    \Big(Q(0) \land \ldots \land Q(n) \to Q(n+1)\Big) \to
    \Big(Q(0) \land \ldots \land Q(n) \to Q(0) \land \ldots \land Q(n+1)\Big)
    \label{eqn:first-second}
  \end{align}
  根据 (\ref{eqn:second-induction-base}), (\ref{eqn:first-induction}) 与 (\ref{eqn:first-induction}),
  我们有
  \begin{align}
    \forall n \in \N. Q(0) \land \ldots \land Q(n) 
  \end{align}
  因此,
  \begin{align}
    \forall n \in \N. \land Q(n)
  \end{align}
  所以, 第二数学归纳法得证。
\end{proof}
%%%%%%%%%%%%%%%%%%%%%%%%%%%%%%